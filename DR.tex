
\documentclass[11pt,a4paper]{book}
\let\cleardoublepage\clearpage

\usepackage[utf8]{inputenc}
\usepackage[english]{babel}
\usepackage[left=2.5cm,right=2.5cm,top=2cm,bottom=2cm]{geometry}
\usepackage{hyperref}
\usepackage{textcomp}
\usepackage{lmodern}
\usepackage{graphicx}
\usepackage{svg}
\usepackage{amsmath}
\usepackage{amsthm}
\usepackage{amsfonts}
\usepackage{amssymb}
\usepackage{multicol}
\usepackage{calc}

\newcommand{\p}{\textquotesingle}
\newcommand{\m}{\texttt}
\newcommand{\ps}{\p\,\,}
\newcommand{\comment}[1]{{\color{gray}\quad//#1}}

\begin {document}

\author{Alexandros Fokianos \& Tommaso Raposio}
\title{A Domino Reduction Guide}


\maketitle 


\section*{A few words before you start reading}

\begin{center}
\textit{The idea is that you make the cube look nicer so it can be solved more easily}\\ 
Sebastiano Tronto, 2019\\
\end{center}
111In the past year the use of Domino Reduction in FMC has grown in popularity among the best at the event and has led to two world records: Harry Savage’s 17 single at TGBBO 2019 and Sebastiano Tronto’s 16 single at FMC 2019. Seeing those incredible solves, and missing out on the two solutions, we (the two authors) have decided to learn this method and want to take you with us in this journey.\\
\newline
There is currently no written documentation about Domino Reduction and the only way you can learn how to do it is by reading other people’s example solves or by asking them to explain you the reasoning behind their choices. This has led to a bit of confusion. There are steps of the method that have different names around the world and there are others that don’t even have one. The purpose of this guide is therefore to offer a more schematic approach to Domino Reduction, uniforming the terminology.\\
\newline
Before you start, we highly recommend you read Sebastiano’s FMC guide (at \url{https://fmcsolves.cubing.net/fmc\_tutorial\_ENG.pdf}), also known as “The Bible” in Italy. This guide's structure will be similar. Most of the terms and acronyms we will be using are common knowledge for regular FMC solvers but if you are new to the event you should start there and get used to them. Sebastiano also provided us with some example solves and helped us during the writing of this guide. Our aim is to cover all we know about Domino Reduction, adding to what he already talked about in the Bible about FMC. Think of this as kind of a “Part 2” of his work, or at least this is what we hope it becomes.\\
As you may have understood, Italian is our first language. So if you find any mistakes or if you think something is explained poorly, feel free to message us directly at our E-mail address at the end of this guide.\\
\newline
We will be using DR from now on to talk about Domino Reduction and we will refer to E-layer edges already in the E layer as "\textbf{good}" edges and non E-layer edges in the E layer as "\textbf{bad}" edges.\\
All the setup cases will be “DR - x E-edges, y corners" and we will call these “x edges, y corners”.\\
As for the example solves, we will be using “c” for corners, “e” for edges and “x” for centers (as suggested by Sebastiano), as well as the most common names for the different stages in the solve e.g. “F2L-1”. So for example, 3c2e2e4x means: “All but 3 corners, a double swap of edges and 4 centers.\\
\newline
DR is easier to learn than you would expect and it requires very little knowledge of FMC to understand the basics of it. But in order to achieve its full potential you need to master a few advanced techniques.\\
To complete a DR you must know how to do an edge orientation. In order to give a full solution with DR you will most likely need to know NISS and corner and/or edge insertions. If instead you want to get a really good DR solve, you either get very lucky or combine the DR finish with Reverse NISS, center insertions, slicey shenanigans, and many other advanced FMC resources.

\tableofcontents

\bigskip
\newpage

\subsection*{FAQ}

\textbf{What is DR?}\\
Domino Reduction is a 2-axis EO with Corner Orientation. You may think of 2-axis EO as a normal EO with all the E layer edges on the E slice. Once the cube has been reduced to a Domino (let’s say on U/D), it can be solved using only the  [U, D, L2 R2, B2, F2] group.\\
\newline
\textbf{Why  should I learn DR?}\\
At the beginning, DR may seem hard to set-up or even harder to “finish” in a reasonable number of moves. On the other hand, mastering it can lead to very nice solves quite consistently.\\
\newline
\textbf{When should I start learning DR?}\\
After you have learned EO you will have all the tools necessary to achieve a DR.
Although if you know only EO as an advanced technique, there are definitely other things you can learn  before DR to improve your solves much more.\\
\newline
\textbf {What do you mean with “after DR the cube feels and looks good”?}\\
You could hear some of the very good FMC solvers say something like this. It is a subjective feeling, that comes mostly with experience. A cube with DR looks easier to solve than a cube without it, in the same way a cube with EO looks easier than a cube without it.\\
DR solves may not be better than any other, but they feel very different and have more room  for creativity.\\
\newline
\textbf{Is DR really that good?}\\
More than 80\%  of combinations have a 10 move or less DR. Some of them are difficult to find but after some practice you can consistently achieve DR in less than 12 moves. Once the DR is done, God’s number for solving the cube is 18 moves *, and this solutions can be found with “human” methods most of the times. This means  that DR is a reliable and consistent method to get sub-30 solves!\\
\footnotesize{*\href{http://kociemba.org/cube.htm}{http://kociemba.org/cube.htm}}


\chapter{Part one:  A Good EO}

The first step of the DR is edge orientation, possibly a “good” one. How do you define a “good EO”?  “Less than 4 moves” or “I have some pairs and blocks” are possible answers but since your aim is to get a DR, we usually consider a good EO the following: 

\begin{center}
\textit{Most corners are oriented (or 1 move away) and some E-edges are already in the E layer}
\end{center}
\bigskip
Let’s take a look at two simple examples:\\


\bigskip
\begin{tabular}{|l|}
\hline
\textbf{First EO - Example }\\
\hline
Scramble: \m{R\ps U\ps F D2 R2 B2 D2 R2 B\ps F\ps L2 U\ps F2 U2 L D U\ps F\ps L\ps D\ps R2 U R\ps U\ps F}\\
\hline
\begin{minipage}[l]{0.650\textwidth}
\bigskip
\m{U\p F \comment{ EO (F/B) (2/2) }}\\
\bigskip
\bigskip
\bigskip
\bigskip
\end{minipage}
\begin{minipage}[c]{0.25\textwidth}
\centering
\def\svgwidth{\columnwidth}
\includesvg[width=\textwidth,height=\textheight,keepaspectratio]{img/FS}
\end{minipage}\\
\hline
\end{tabular}
\bigskip
\newline
An F/B EO can lead to a DR with oriented corners on U/D or R/L.
If you do a z rotation (this preserves EO), you will see that you already have 4 oriented corners and 3 good edges. This is a promising start for a domino reduction.\\
If you look at the F/B EO without the z rotation, you only have 2 oriented corners and 2 good edges. \\
It’s very important that you check both possible orientations of the cube after the EO. Although there are only 3 possible DRs: U/D, R/L, F/B you can get to each one of them from two different EO's. Keep this in mind or you'll miss 3 starts out of the possible 6 (50\% is a lot!)\\


\bigskip
\begin{tabular}{|l|}
\hline
\textbf{Second EO - Example }\\
\hline
Scramble: \m{R\ps U\ps F L2 B2 R2 U2 F D2 B\ps L2 F\ps D2 U B L\ps U2 F\ps R2 B2 U\ps R\ps D R\ps U\p}\\ \m{F}\\
\hline
\begin{minipage}[l]{0.650\textwidth}
\bigskip
 \m{B\ps R\ps D\ps L\ps \comment{ EO (R/L) (4/4) }}\\
\bigskip
\bigskip
\bigskip
\bigskip
\bigskip
\end{minipage}
\begin{minipage}[c]{0.25\textwidth}
\centering
\def\svgwidth{\columnwidth}
\includesvg{img/SS}
\end{minipage}\\
\hline
\end{tabular}
\bigskip
\newline
Even though there are only 4 oriented corners on U/D and two bad edges, we are 1 move away from a very promising case. 
With F or F’ we go from 2 bad edges to 1 and from 4 oriented corners to 5 (4-1+2).\\
Try to play around with these two starts and see if you can orient all corners or get 4/4 good edges in a few moves. Once you’re convinced it is a difficult task without a hint, you can move on to the next part.

\chapter{Part Two: Orienting Corners and Getting Good Edges}

Once you have found a good EO start (remember there could be none), the next step is the actual domino reduction.  
The best approach to get all 4 good edges and the CO is to try and solve these two problems at the same time, saving moves.
The are two sets of cases that can occur. If you have to deal with both E-edges and corners we call these\textbf{ Domino Reduction Triggers}. If you have to deal with just corners or E-edges or you have both but EO is missing, we call these \textbf{Partial Domino Reduction}.

\section{DR Triggers}

Let’s look at the most simple one:
Take a solved cube and apply R U2 R’. 
If you look at the cube you can see that you have 4 misoriented corners and only 1 bad edge. From this position, you can complete the DR in various ways, but the shortest ones are:
\bigskip
\begin{center}
R U2  R or R U2  R\p \\
L F2  L  or L F2  L\p
\end{center}
\bigskip
You can easily remember this case by looking at the position of the U/D stickers: the ones near the U/D edge form a “bar”, the ones near the E-layer edge form a “column”.
Here are all of the most useful triggers we know of, assuming the EO is done on F/B with white on top.\\

\bigskip
\begin{tabular}{|l|}
\hline
\textbf{2 Corners, 1 Edge - WV }\\
\hline
\begin{minipage}[l]{0.650\textwidth}
\bigskip
\bigskip
\m{L\ps U\ps R U L R\ps \\
L R U\ps R\ps U L\\
Other Variations\\}
\bigskip
\bigskip
\end{minipage}
\begin{minipage}[c]{0.25\textwidth}
\centering
\def\svgwidth{\columnwidth}
\includesvg{img/WV}
\end{minipage}\\
\hline
\end{tabular}
\bigskip
\newline
These kinds of triggers are very difficult to spot and setup in few moves.
When you find yourself in a 2c1e case
you may want to execute some moves to get to another case.\\
\newpage
\bigskip
\begin{tabular}{|l|}
\hline
\textbf{3 corners, 1 edge - F2L }\\
\hline
\begin{minipage}[l]{0.650\textwidth}
\bigskip
\bigskip
R U R or R U R\p \\
R U' R or R U' R\p\\
L\ps U L or L\ps U  L\p\\
L\ps U' L or L\ps U' L\p\\
\bigskip
\bigskip
\end{minipage}
\begin{minipage}[c]{0.25\textwidth}
\centering
\def\svgwidth{\columnwidth}
\includesvg{img/F2L}
\end{minipage}\\
\hline
\end{tabular}
\bigskip
\newline
These can be seen as simple F2L cases.\\

\bigskip
\begin{tabular}{|l|}
\hline
\textbf{4 corners, 1 edge - Column \& bar }\\
\hline
\begin{minipage}[l]{0.650\textwidth}
\bigskip
\bigskip
L\p U2 L or L\p U2 L\p \\
R\p F2 R or R\p F2 R\p \\
R U2 R or R U2 R\p \\
L F2 L or L F2 L\p\\
\bigskip
\bigskip
\end{minipage}
\begin{minipage}[c]{0.25\textwidth}
\centering
\def\svgwidth{\columnwidth}
\includesvg{img/CB}
\end{minipage}\\
\hline
\end{tabular}
\bigskip
\newline
These are the ones from the example. To set them up, focus on making the bar in front and the column on top at the same time.\\

\bigskip
\begin{tabular}{|l|}
\hline
\textbf{ 4 corners, 2 edges - Conjugate and Solve }\\
\hline
\begin{minipage}[l]{0.650\textwidth}
\bigskip
\bigskip
R or R\p \\
L or L\p
\bigskip
\bigskip
\end{minipage}
\begin{minipage}[c]{0.25\textwidth}
\centering
\def\svgwidth{\columnwidth}
\includesvg{img/CS}
\end{minipage}\\
\hline
\end{tabular}
\bigskip
\newline
If you have all the pieces in the correct position on the same face, DR can be achieved with a single move. Focus on making two columns with the E edges on U/D  and bring them on the same layer. Pay attention to U/D edges to be on the same layer too!\\
All the other combinations of number of bad edges and misoriented corners lead to one of the above with a few moves. \\
\newline
The best way to learn the triggers is to figure out a personal way to recognize them and play around with them a little bit. 
During your first tries, take a second solved cube and use it to take some DR triggers as reference and see if you can find the way to set them up on your scrambled one.\\
At first you could find this very hard. Here are some examples to help you figure out the reasoning behind achieving DR. 

\bigskip
\begin{tabular}{|l|}
\hline
\textbf{First EO - Example }\\
\hline
Scramble: \m{R\ps U\ps F D2 R2 B2 D2 R2 B\ps F\ps L2 U\ps F2 U2 L D U\ps F\ps L\ps D\ps R2 U R\ps U\ps F}\\
\hline
\begin{minipage}[l]{0.650\textwidth}
\bigskip
 \m{U\ps F \comment{ EO (F/B) (2/2) }\\
(R B2 L2 B2) \comment{ setup to trigger (4/6) }\\
(D\ps L2 D\ps) \comment{ DR trigger (3/9)} }\\
\bigskip
\bigskip
\end{minipage}
\begin{minipage}[c]{0.25\textwidth}
\centering
\def\svgwidth{\columnwidth}
\includesvg{img/FS}
\end{minipage}\\
\hline
\end{tabular}
\bigskip
\newline
U' F // EO (F/B)\\
Use NISS and look at the inverse scramble. With a z rotation you can easily see you are in a 4 corners 1 edge case (the orange-blue edge should not be in the E layer).  
Your aim is to set-up the Column \& Bar case: the bar with the orange-blue edge, the column with the white-blue edge.\\
\newline
(R B2 L2 B2) // setup to trigger\\
\textit{R - joins an orange sticker with the L/R edge forming a semi-bar}\\
\textit{B2 - joins a corner with the E edge to form a semi-column}\\
\textit{L2 - completes the bar}\\
\textit{B2 - completes the column}\\
\newline
Now most of the work is done. Notice how moves of the U/D layers (in this case R/L) and double moves of the other layers don’t affect the number of oriented corners or of good edges\\
\newline
(D' L2 D’) // DR trigger\\
We could have used a variation of the trigger, e.g. U F2 U. \\
Ideally you should check every option  but considering you only have one hour in official attempts, you should choose the trigger that leaves you with the most blocks or with easy to solve corners. The second one is crucial in DR finishes, but we’ll get to that later.
There are two other ways of achieving DR from that EO on the normal scramble that use the same concept of this setup.
\begin{multicols}{2}
\hfill \break
U’ F //EO \\
R2 L2 U2 L // setup \\ 
\columnbreak
U L2 U’ // trigger\\
\hfill \break
U’ F //EO \\
R’ L B2 R // setup\\
D L2 D’ // trigger
\end{multicols}
\hfill \break
These examples make use of a double move (U2 in the first one and B2 in the second one) to both setup some corners and move the bad edge in the correct position. Before this double move, the position of the edge would not have worked with the corners to trigger the DR. \\

\bigskip
\begin{tabular}{|l|}
\hline
\textbf{Second EO - Example }\\
\hline
Scramble: \m{R\ps U\ps F L2 B2 R2 U2 F D2 B\ps L2 F\ps D2 U B L\ps U2 F\ps R2 B2 U\ps R\ps D R\ps U\p}\\ \m{F}\\
\hline
\begin{minipage}[l]{0.650\textwidth}
\bigskip
\m{B\ps R\ps D\ps L\ps\comment{ EO (R/L) (4/4) }\\
F or F\p \comment{ 4c2e -> 3c1e (1/5)}\\
(D\p F2)\comment{ setup (2/7) }\\
(B\p U\p B\p)\comment{ DR trigger (3/10) }}\\
\bigskip
\end{minipage}
\begin{minipage}[c]{0.25\textwidth}
\centering
\def\svgwidth{\columnwidth}
\includesvg{img/SS}
\end{minipage}\\
\hline
\end{tabular}
\bigskip 
\newline
B' R' D' L'  // EO \\
F /  F’ // 4c2e to 3c1e\\
\newline
Both F and F' end up working for our DR setup.
Look at the inverse with NISS. Your aim is now to try and setup one of the F2L cases (3c,1e). To get a better idea of what you are going  for, apply L’ U L on a solved cube.\\
(D' F2) // setup\\
\textit{D’ - moves a yellow sticker to form a semi-bar with U/D edge}\\
\textit{F2 - moves the last corner to form a semi-column with E edge}\\
\newline
(B' U' B') // DR trigger\\
DR is done!\\
\newpage
\section{Partial Domino Reduction}

Sometimes, after you complete a DR step you may get what we call a Partial Domino Reduction, or PDR. It’s “partial” because you already have two of the three steps needed for DR, solved: CO, E-layer or EO. In this section we will go through all the possibilities.\\
Remember that a perfectly valid option with PDR is to keep going with the solve as if you had DR and fix everything with insertions at the end.\\

\subsection{\textit{PDR in which you have CO and EO}}

\textbf{1 bad edge}\\
There are two ways to deal with this case. The first one is to insert a commutator, either during the solve or at the end when you have a full skeleton. Edge commutators may cancel more than average in DR solves so it would be better to leave this at the end. Nonetheless, if you decide to insert the commutator while setting up the trigger, then you must pay attention to what edges you are cycling. For a  3 cycle the only valid options are:\\
\newline
Bad edge → E-edge in the U/D layer → Good edge\\
Bad edge → U/D edge → E-edge in the U/D layer \\
\newline
The other option is to insert a slice that swaps the bad edge with the last E-edge and either two U/D edges or two good edges. Here is an example of this by Sebastiano Tronto:\\

\bigskip
\begin{tabular}{|l|}
\hline
\textbf{First PDR Example }\\
\hline
Scramble: \m{R\ps U\ps F U\ps F2 U\ps F2 L2 U\ps B2 D2 F U2 B\ps D\ps B D\ps F2 R\ps U\ps L\ps B U2 R\ps U\p}\\ \m{F}\\
\hline
\begin{minipage}[l]{0.650\textwidth}
\bigskip
\m{F U D\ps R\comment{ EO (4/4)}\\
F \comment{ 3e1c (1/5)}\\
L2 U\ps * B U\ps B\ps \comment{1e (5/10)}\\
* = U\ps M2 U (3/13) \comment{ DR}}
\bigskip
\end{minipage}
\begin{minipage}[c]{0.25\textwidth}
\centering
\def\svgwidth{\columnwidth}
\includesvg{img/PDR1}
\end{minipage}\\
\hline
\end{tabular}
\bigskip
\bigskip
\newline
\textbf{2 bad edges}\\

\bigskip
\begin{tabular}{|l|}
\hline
\textbf{Break \& Solve }\\
\hline
\begin{minipage}[l]{0.650\textwidth}
\bigskip
\bigskip
\m{R\ps E2  R or R E2 R'\ps\\
Other Variations \\}
\bigskip
\bigskip
\end{minipage}
\begin{minipage}[c]{0.25\textwidth}
\centering
\def\svgwidth{\columnwidth}
\includesvg{img/BSS}
\end{minipage}\\
\hline
\end{tabular}
\bigskip
\newline
Your goal is to form a column with the E edges on the opposite face as the 2 bad edges and then swap them with R’ E2 R and variations.
As in the previous case, you can insert a slice move in your solve that swaps the 2 bad edges with the two E-edges. R’ U2 D2 L = R’ E2 R is in fact \bigskip
the easiest case of setup to a single slice.\\
\newpage
\hfill \break
\textbf{3 or 4  bad edges}\\
This is not a very good case but you have a few options to consider:\\
-You could insert two slices that swap the bad edges with E-edges.\\
-You could perform a single move that takes you either to a 4c1e case or to a 4c2e case.\\
-You could keep on going with your corners first solve and insert the edges when you have a full skeleton.\\
\newline
If none of these seem to have good continuations, you should start looking for a different DR
\bigskip
\subsection{\textit{PDR in which you already have E-layer and EO}}

\textbf{2 bad corners}\\

\bigskip
\begin{tabular}{|l|}
\hline
\textbf{Fake Swap }\\
\hline
\begin{minipage}[l]{0.650\textwidth}
\bigskip
\bigskip
\m{R U\ps L2 U R\ps \\
Other Variations \\}
\bigskip
\bigskip
\bigskip
\end{minipage}
\begin{minipage}[c]{0.25\textwidth}
\centering
\def\svgwidth{\columnwidth}
\includesvg{img/2BC}
\end{minipage}\\
\hline
\end{tabular}
\bigskip
\newline
Like the 1e case, this can also be tamed inserting a commutator either in the setup or in the skeleton at the end of the solve.\\
\bigskip
\newline
\textbf{3 bad corners}\\
\newline
You can insert a sune or a 3-corner commutator. Keep in mind that the sune does a 3 cycle of edges as well, so the same rules that were explained in the "1 bad edge" section for which edges can be cycled apply here.\\
If you decide to insert a commutator during the setup phase, you should go for the classic 8movers (pure). If you decide to keep on solving with PDR and insert the corners after you have a full skeleton, in section 5.1 we will talk about "DR special" algorithms that may cancel more moves.\\
\bigskip
\newline
\textbf{4 to 8 bad corners}\\
It's better to look for another DR\\
\bigskip


\subsection{\textit{PDR in which you already have CO and E-layer}}

\textbf{0 corners and 0 edges}\\
You can use M’ U M, M’ U M’ and other Roux-like EO’s.\\
In this particular case you already have 4 good edges and all the corners oriented but you don’t have EO. To achieve DR you need to orient the edges using the S and M slices like in the second to last step of the Roux method\\
\newline
Alexandre Campos has a nice \href{https://docs.google.com/document/d/1oZwr2aSllFBL5lhbLTiWKQWplfk4i0LN0wA0uskeLJs/edit?usp=sharing}{document (click here!)} that collects a lot of corner-PDR and DR solves


\chapter{Part Three: Alternative Approaches}

This guide will be mainly focused on the three-step method described in the previous sections: EO first, then setup to DR trigger and finally the trigger itself. This method is the most common and probably the most reliable to get a DR within 14 moves. With 6 different possibilities (2 for each EO) that become 12 if you also check the inverse with NISS, there is a high chance to get a decent solve. This doesn’t mean you should use DR in every attempt though. With practice you will get fast enough to spot triggers in a reasonable amount of time. This allows you to add a “DR check” in your FMC routine. Try to apply DR whenever possible, but don’t try to force it.\\
\newline
If, instead, you want to focus mainly on DR, here is some advice:\\
\newline
-Try keeping track of corner orientation doing EO, possibly influencing it.\\
-Pairs or pseudo pairs, made by joining same or opposite colours together, can lead to a better trigger case \\
-Each EO has two possible endings, try both (e.g. F/F’)\\
-Try influencing E-layer edges during EO, you should end up with at least 2 good edges to get a good DR.\\
\newline
Different approaches consist in joining some of the steps together or changing their order. In the next sections we will talk about them very briefly since a lot of this stuff is still unexplored.


\section{Orient First}

In this paragraph we’ll show you some solves by Attila Horváth. He applies DR orienting corners first and then orienting and fixing edges. 
Orienting corners at the start of a solve can be tricky to learn for most cubers If you want to practice, scramble a 2x2 cube and try to orient all the pieces in the fewest moves possible. A few ideas to start are:\\
\newline
- Try to create corner bars and simultaneously corner columns.\\
- Understand how moves are going to affect the orientation of corners.\\
- Don’t bother about the position of edges or centers.\\
- Study as much example solves as possible.\\
- Remember that sometimes orienting corners can take lots of moves and it’s not always efficient.\\
\newline
Doing is better than reading. Follow along this solve by Attila!

\bigskip
\begin{tabular}{|l|}
\hline
\textbf{First CO Example }\\
\hline
Scramble: \m{R\ps U\ps F R2 B\ps R2 B U2 B2 F\ps R2 B2 U B U2 R D B\ps U B2 L2 F L\ps F2 R\ps}\\ \m{          U\ps F}\\
\hline
\begin{minipage}[l]{0.650\textwidth}
\bigskip
\m{(F R2 U R)\comment{ orient R/L corners (4/4)}\\
(D2 R\ps U D\ps F) \comment{ orient edges to PDR (5/9) }\\
R L2 D\ps R2 L2 U \comment{ DR ( 6/15) }\\
F2 R2 L2 D2 R\ps B2 R2 \comment{ Solved (7/21) }\\
}
\bigskip
\bigskip
\end{minipage}
\begin{minipage}[c]{0.25\textwidth}
\centering
\def\svgwidth{\columnwidth}
\includesvg{img/CO1}
\end{minipage}\\
\hline
\m{Solution: R L2 D\ps R2 L2 U F2 R2 L2 D2 R\ps B2 R2 F\ps D U\ps R D2 R\ps U\ps R2 F\ps (21)}\\
\hline
\end{tabular}
\hfill \break
\newpage
Explanation:
\newline
\textit{F - creates a bar and a column\\
R2 - creates a second bar and second column\\
U R - orientates corners}\\
\newline
Remember that at the beginning, the position of the centers is not relevant: we want to orient corners in relation to each other.
Once you have CO, the next step should be EO. This can be done in multiple ways, using slices to move edges around or breaking and fixing corners. As always, our aim is to solve edges in the most efficient way possible. 
If you orient edges with slices, you’ll find that edge skeletons are much easier to insert.\\
\newline
(D2 R' U D' F) \comment{orient edges to PDR}
\newline
This is a setup to E slice. It can be hard to see at first, but rules are pretty much the same of normal EOs. The only difference is that doing half turns affects corner orientation. That’s why you must pay attention to it and always perform the anti-setup to fix it. Also remember that centers are not relevant before EO but they are AFTER.\\
\newline
R L2 D' R2 L2 U \comment{ DR}
\newline
DR cases for orienting first are tricky. Here we have the basic one: 2 edges, no corners. You can solve this with the Break and Solve DR Trigger\\
\newline
F2 R2 L2 D2 R' B2 R2 \comment{ Solved }
\newline
Solving the cube after an orient corners first DR can be very easy, since we can go back and influence the corners’ solution and/or make blocks  moving edges with slice to get a linear ending. This is difficult enough to do and requires practice.\\
\newline
There are three main advantages of using this method:\\
\newline
- Once you choose one of the three possible CO, you’ll have only one DR to check.\\
- You can immediately see if the corners are easy to solve.\\
- It’s very easy to influence edges during the CO steps using slices.\\
\newline
This method is at his best when you realise you can modify the CO a little to orient some edges and fix the E layer. The DR should then be a few moves away.\\
\hfill \break
These are two more example solves. As always, try to follow bars’ and columns’ creation during corner orientation. \\


\begin{tabular}{|l|}
\hline
\textbf{Second CO Example }\\
\hline
Scramble: \m{F R\ps F2 U\ps L F R F\ps D B\ps D\ps L2 B2 U F2 R2 U\ps L2 B2 U B2}\\
\hline
\begin{minipage}[l]{0.650\textwidth}
\bigskip
\m{R L\ps F B\ps L\ps U\ps R\ps B L\ps\comment{ Orient R/ L corners (9/9) } \\
Note that R L\ps F B\ps are useless to CO.\\
F L\ps F B\ps U R2 F B\ps U2 L U D\ps B\ps \comment{ DR ( 13/22) }\\
R2 F2 L F2 L U2 L\ps R2 \comment{ Finish ( 8/30) }
}
\bigskip
\bigskip
\bigskip
\bigskip
\end{minipage}
\begin{minipage}[c]{0.25\textwidth}
\centering
\def\svgwidth{\columnwidth}
\includesvg{img/CO2}
\end{minipage}\\
\hline
\m{Solution: R L\ps F B\ps L\ps U\ps R\ps B L\ps F L\ps F B\ps U R2 F B\ps U2 L U D\ps B\ps R2 F2 L F2}\\
\m{ L U2 L\ps R2(30)}\\
\hline
\end{tabular}
\bigskip
\newline
Attila’s method is mainly focused on influencing the next step while doing the previous one. You can see the first step above as:\\
\newline
* B\ps L\ps F\ps U B\ps \comment{ CO }\\
* M\ps S\ps \comment{ Influences edges orientation}\\
\newline
Again, in this steps Attila inserts lots of slices to move edges around. This has two main reasons: getting to DR and creating blocks. Try to follow along ignoring centers position.\\
\newline
F ° U2 *  D2 £ B2  L2  F L2 F D2 F ! \comment{ solves corners, leaves 6edges} \\
\textit{° : L\ps S\ps L - set-up to slice to moves some E-edges}\\
\textit{* : S\ps  - set-up ( moves an R/L edge)}\\
\textit{£ : R E  R\ps - set-up to slice to orient last edges}\\
\textit{ ! : S2 - solves last 4edges,4centers}\\
\bigskip
\newline
Next solve is by Sebastiano Tronto:

\bigskip
\begin{tabular}{|l|}
\hline
\textbf{Third CO Example }\\
\hline
Scramble: \m{R\ps U\ps F U R2 D R2 B2 D\ps B2 L\ps D2 L2 R D U L2 F\ps U2 B L\ps B R\ps U\ps F}\\
\hline
\begin{minipage}[l]{0.650\textwidth}
\bigskip
\m{U\ps B\ps U B \comment{ CO, corners 2 moves from solved (4/4)}\\
( D2 L F\ps R\ps L D)\comment{ DR (6/10) }\\
(R L\ps) \comment{ 3e2e2e (2/12) }\\
Skeleton 1: U\ps B\ps U B * R\ps L D\ps L\ps R F L\ps D2 \\ (12)
* = B2 L2 B2 R2 F2 R2 \comment{ 2e2e (6-4/16)}}
\bigskip
\end{minipage}
\begin{minipage}[c]{0.25\textwidth}
\centering
\def\svgwidth{\columnwidth}
\includesvg{img/CO3}
\end{minipage}\\
\hline
\end{tabular}
\bigskip
\newline
Replace the last 7 moves with D\ps R L\ps F D2 to get a 5e case:\\
\newline
Skeleton 2: U\ps B\ps U B\ps * L2 B2 R2 + F2 R D\ps R L\ps F D2 \\
* = E\ps \\
+ = [R\ps B\ps D\ps B: E]\\


\section{Sledge/Hedge DR}

Another method to obtain DR avoiding EO was suggested by Chong Wen but we couldn’t see it applied anywhere in any solve. Basically, this trick consists of using DR Triggers that influence EO too, thus saving some moves\\
\newline
R’ F R F’ – 2 non-oriented U/D edges; 1 oriented E-edge; 4 corners\\
F R’ F’ R - 2 non-oriented edges, one U/D and one E; 1 oriented U/D edge; 4 corners\\
\newline
As always, execute the inverse of this triggers to understand how corners should be positioned.\\
We think that practicing this is not worth it but it’s nice to know and think about it nonetheless. In our opinion it can be useful only if you find yourself in the perfect situation. Scrambles with only 2 misoriented edges and half of the corners oriented are not so rare, although most of times it’s convenient to just do EO.\\
\newline
Here’s an attempt made by one of the authors (Alexandros Fokianos)\\

\bigskip
\begin{tabular}{|l|}
\hline
\textbf{Sledge/Hedge Example }\\
\hline
Scramble: \m{R\ps U\ps F R2 D\ps L2 U\ps F2 R2 F2 L2 U\ps F2 D F L2 U2 R F\ps D2 F U\ps L D U}\\ \m{ R\ps U\ps F}\\
\hline
\begin{minipage}[l]{0.650\textwidth}
\bigskip
\m{(R' L D' L)\comment{ Orienting stuff (4/4) }\\
(B2 D B2 U2 B2) \comment{ Setup (5/9)}\\
(B L' B' L')  \comment{ DR trigger (4/13) }}
\bigskip
\end{minipage}
\begin{minipage}[c]{0.25\textwidth}
\centering
\def\svgwidth{\columnwidth}
\includesvg{img/SH}
\end{minipage}\\
\hline
\end{tabular}
\bigskip
\newline
Setup explanation:\\
\textit{B2 D - a column with E-edge is already formed. B2 D form a semi-bar with the misoriented U edge}\\
\textit{B2 U2 B2 - simply setups the case. To follow this, keep the cube with blue front and white up.}\\
\newline
It took me some moves in the beginning to correctly orient pieces to get the sledge case. Setup is not that hard but keep an eye for conveniently oriented scrambles. DR could be 7 or 8 moves away and this method is more efficient than doing EO for just two edges.\\
\newpage
\section{E Slice First}

 There is another way of getting to DR mixing the order of the steps.\\
This method is a bit on the difficult side but can become very powerful if used on the right scramble.\\
\newline
First, we fix the E layer, orienting some corners along the way. This becomes very easy if you already have 2 or 3 good edges. Avoiding EO makes things a lot easier since you have no move limitations. The next step is CO. You can make use of the DR triggers we learned at the begging to move around the last E edge orienting corners.\\
At this point you have a PDR that’s missing EO. You can apply some Roux methods to get DR. In the best case, you will have only 4 misoriented edges that can be solved using M’ U M triggers. Even if using slices may seem inefficient, it can be useful later during edge insertions.\\
\newline
If you choose to apply this method, here are some tips:\\
-Try to influence EO while doing CO \\
-Play around with E-edges and choose different corners to orient\\
\newline
Here is an example:\\

\bigskip
\begin{tabular}{|l|}
\hline
\textbf{E-First Example }\\
\hline
Scramble: \m{L\ps U B\ps D2 F\ps D\ps F2 R L2 D\ps F2 R2 D2 F2 L2 U\ps L2 D2 B2}\\
\hline
\begin{minipage}[l]{0.650\textwidth}
\bigskip
\m{U2 D F \comment{ 3 E-edges + some CO U/D (3/3)}\\
D\ps U\ps L\ps D\ps L \comment{ CO + last E-edge (5/8)}\\
L R\ps F R2 B L R\ps \comment{ EO trigger to DR (7/15) }}
\bigskip
\end{minipage}
\begin{minipage}[c]{0.25\textwidth}
\centering
\def\svgwidth{\columnwidth}
\includesvg{img/EF}
\end{minipage}\\
\hline
\end{tabular}
\bigskip
\newline
Setup explanation:\\
\textit{D’ U’ – setups the case}\\
\textit{L’ D’ L – DR trigger (except it doesn’t make DR here)}\\

\chapter{Part Four: Solving Methods}

After you reduce the cube to a domino, you are not done yet! The last step of the DR solve, the “finish”, is where all the magic happens.
You can imagine the DR as a solid tree trunk with different branches: CF, Slice insertions, HTR reduction and many more but for each solution only one of these has the golden apple hanging at the end of it.
In this part of the guide we’re going to present you the basics of the most used methods to finish a DR. Once again, examples are the best way to learn.\\
\newline
To get started, you may want to get your hands on a Domino Cube (2x3x3). This can be solved with only U, D, L2, R2, B2, F2 moves, and is exactly the reason for the name Domino Reduction.
Solving the Domino cube is a good way of practicing finishes although we don’t suggest using domino cube algorithms since they don’t work well for FMC.\\

\bigskip
\section{Method 0 - Normal Skeletons}

These are the type of skeletons you will get from “normal” attempts. The only difference is that with DR you can expect edge insertions to cancel a bit more moves than average.  This is due to the high number of opposite moves in the skeleton and because, in general, the best edge insertion algorithms (e.g. 6 move 3e cycles, or easy 2e2e) already have EO.\\

\bigskip
\begin{tabular}{|l|}
\hline
\textbf{Normal Skeleton }\\
\hline
Scramble: \m{R\ps U\ps F D\ps B2 R2 U\ps F2 D\ps F2 U B2 R2 F2 L\ps B F\ps R2 D L F2 U R\ps D2 R\ps}\\ \m{U\ps F}\\
\hline
\begin{minipage}[l]{0.650\textwidth}
\bigskip
\m{(B\ps) U\ps D2 B \comment{ EO (4/4) }\\
D\ps L2 F2 U\ps \comment{ Setup (4c2e case) (4/8) }\\
L\ps \comment{ DR trigger (Conjugate and solve) (1/9) }\\
F2 D2 L2 (L2) \comment{ Blocks (4/13}\\
R2 U F2 U2 \comment{3c4e (4/17) }\\
Skeleton: U’ D2 B D\ps L2 F2 U\ps L\ps F2 D2 L2 R2 U F2 U2 L2 ° B\\
° L2 U2 R2 D2 R2 £ U2 (6-4/19) \\
£ R2 U\ps L2 U R2 U\ps L2 U (8-3/24) \\
}
\bigskip
\end{minipage}
\begin{minipage}[c]{0.25\textwidth}
\centering
\def\svgwidth{\columnwidth}
\includesvg{img/NS}
\end{minipage}\\
\hline
\m{Solution: U\ps D2 B D\ps L2 F2 U\ps L\ps F2 D2 L2 R2 U F2 R2 D2 U\ps L2 U R2 U\ps}\\ \m{L2 U\ps B (24)}\\
\hline
\end{tabular}
\bigskip
\newline
Setup explanation:\\
D\ps L2 F2 U\ps \comment{ setup (4c2e case)}\\
\textit{D\ps  L2 - forms a column with the red-green edge}\\
\textit{F2 - brings one U edge on the L layer and completes second column with the red-blue edge}\\
\textit{U\ps - brings both columns on L layer}\\
\newline
Now Let’s get to the exciting stuff\\

\bigskip
\section{Method 1 - Corners First}

Attila Horváth usually starts his solves orienting and solving corners while trying to solve as many edges as possible. This technique can be good for DR as well if the corners are easy to solve. \\
\newline
You can find more info about corner first as a method here: \url{https://www.speedsolving.com/wiki/index.php/Corners\_First}\\
\newline
This approach can be very good for DR starts since corner insertions can be very inefficient. Most of the times you will find solutions for corners that influence the edges in different ways, be sure to take advantage of that. Also note that you don’t necessarily have to solve corners relative to centers but only to each other. The downside of this method is that solving corners will be sometimes very difficult.\\

\bigskip
\begin{tabular}{|l|}
\hline
\textbf{CF example }\\
\hline
Scramble:\m{R\ps U\ps F L2 B2 R2 U2 F D2 B\ps L2 F\ps D2 U B L\ps U2 F\ps R2 B2 U\ps R\ps D R\ps U\ps}\\ \m{F}\\
\hline
\begin{minipage}[l]{0.650\textwidth}
\bigskip
\m{B\ps R\ps D\ps L\ps \comment{ EO (4/4) }\\
F’ (D\ps F2) \comment{ Setup (3/7) }\\
(B\ps U\ps B\ps) \comment{ DR trigger (3/10) }\\
L2 U B2 D\ps R2 U\ps D2 \comment{ Solves corners and leaves 5e (7/17)}\\
}
\bigskip
\end{minipage}
\begin{minipage}[c]{0.25\textwidth}
\centering
\def\svgwidth{\columnwidth}
\includesvg{img/SS}
\end{minipage}\\
\hline
\end{tabular}
\bigskip
\newline
Many world class solvers try to solve corners as soon as they get a good enough DR start. If corners are easy to solve then you can focus on finding nice insertions for edges: as Sebastiano Tronto once said: “DR + easy corners = insert right away!”. As I said before, you can expect these to cancel more than average on edge skeletons because of opposite moves.\\
There isn’t a specific method to consistently solve corners in few moves, it’s really scramble dependent. Some tips we can give you are:\\
\newline
- Study some CF-users solves to learn some tricks.\\
- Try to start solving two opposite pairs of corners together. If you’re lucky enough, the other 4 will be easy to solve.\\
- Scramble a 2x2 and practice with it (remember that you’re not allowed to use it in competition).\\
- If you know some short CP algorithms you may find them useful in specific occasions.\\
\newline
You'll find lots of CF solves in this document too.\\

\bigskip
\section{Method 2 - Block building and linear endings}

A linear ending is a DR finish that relies mainly on intuition. You first start by building some blocks and you finish…by building other blocks. Linear endings may seem the easiest to master but it turns out that solving a Domino cube using only your intuition is not a simple task. That’s because there isn’t a standard approach to linear endings. You just “put pieces together” to solve the cube. Here are some tips:\\
\newline
- As usual, try to make squares or a 2x2 as a starting point. \\
- When you set up your DR trigger, keep an eye for possible squares/blocks to form. It’s also possible to start the solve with some random blocks + EO and then -setup the DR.\\
- Avoid diagonal corner swaps while making blocks.\\
- Try to influence the E layer while moving around pairs and pieces. This can make the difference towards the end of the solution.\\
- After you’ve made nice blocks, the nice ending should be easy enough to see. If you can’t find it in one or two minutes, go back and change your blocks.\\
- Speed is key. As you get faster to find DR’s, you should get faster in solving blocks in different ways. This would often lead to a nice ending.\\
- Try to make blocks un U/D layers and avoid solving E-layer.\\
- Be creative!\\
- Sometimes destroying already existing blocks that are in the way, can lead to a better block-building in the end.\\
\newline
Block-building is not an exact science, and this is true not only for DR. Usually block building can turn out to be very frustrating but with DR things are a bit easier. Try to “spam finishes” and quit only when you are really convinced your start has no good continuation. \\
\newline
Here is a solve by Sebastiano Tronto in which he uses this type of finish. You can get the importance of the tips above and you can feel the power of a good DR start.This is the same scramble in which Harry Savage got his 17 WR single. We’ll look at his solve in the next section.


\bigskip
\begin{tabular}{|l|}
\hline
\textbf{BB example }\\
\hline
Scramble: \m{R\ps U\ps F D\ps L2 B2 R2 B2 U F2 D U F2 R2 F D R2 B L D\ps B2 R\ps D\ps F2 R\ps}\\ \m{ U\ps F}\\
\hline
\begin{minipage}[l]{0.650\textwidth}
\bigskip
\m{R L\ps D\ps L \comment{ EO (4/4)}\\
(R2 F) \comment{ DR, 2e-4c case (2/6) }\\
(L2 D R2 U2 R2)  \comment{ Same blocks (5/11) }\\
(U2 D R2 D B2 U\ps D2)  \comment{ Finish (7/18) }\\
}
\bigskip
\end{minipage}
\begin{minipage}[c]{0.25\textwidth}
\centering
\def\svgwidth{\columnwidth}
\includesvg{img/BB}
\end{minipage}\\
\hline
\m{Solution: R L\ps D\ps L D2 U B2 D\ps R2 D\ps U2 R2 U2 R2 D\ps L2 F\ps R2 (18)}\\
\hline
\end{tabular}
\bigskip
\newline
Explanation:\\
R L\ps D\ps L \comment{ EO }\\
(R2 F) \comment{ DR, 2e-4c case}\\
\textit{ R2 - setups the columns on F layer}\\
\textit{ F - triggers the domino}\\
\newline
This Domino Reduction is clearly very lucky and “easy”. Sebastiano said he found it by accident as he was just trying to form a square. Finding starts like this is not very common, in fact not every solve is a world record.\\
\newline
• First finish (19):\\
(L2 U F2 U\ps D\ps L2)  \comment{blocks }\\
The general idea is to try to influence the E layer in a good way, while making some blocks. There are multiple options to make blocks on U/D layers. Ideally you should try them all and find the one that gives you the finish. After some time, you’ll get very fast at it.\\
\newline
(U D2 F2 D R2 D\ps U2)  \comment{ finish }\\
Once there are some blocks, it’s time to put them together. Follow the E-edges movement and focus un U/D blocks, to goal is to solve both at the same time.\\
\newline
• Second finish (18):\\
(L2 D R2 U2 R2)  \comment{ same blocks }\\
Here, Sebastiano goes back to his solution and tries to improve it.  He manages to make the same blocks again, influencing the E layer in a different way, but with one move less!\\
\newline
(U2 D R2 D B2 U\ps D2)  \comment{ finish }\\
The ending is technically very similar to the first one.\\
\newline
Comment by Sebastiano:\\
“At first I used (R2 F') to make that extra square, but I could not finish the domino solve from there. Changing F' to F gave an easier (in my opinion) but sub-optimal solve. I found the 18 in about 20 minutes total. The two solutions that I have found are the same if you disregard the E layer; I have tried improving it further with the same strategy, possibly leaving some E-layer edges unsolved and inserting them, without success."\\

\section{Method 3 - HTR solves}
HTR, also referred to as H2, is the third step of the Kociemba algorithm and it consists of reducing the cube to a state solvable within the [ U2, D2, F2, B2, L2, R2 ] group. \\
\newline
If you visit Ryan Heise’s site \href{ https://www.ryanheise.com/cube/human\_thistlethwaite\_algorithm.html}{(click here!)}, you’ll understand that H2 is a very important step of the Human Thistlewaite Algorithm (HTA) as well, the human way of doing Kociemba. This case has been called in many different ways such as Double Domino, Domino on 2 axis or G3. For the purpose of FMC we’ll call it Half Turn Reduction or HTR.\\
It’s fairly simple to understand how to get to this state. As you did for DR, where you had only white/yellow stickers on the U/D layers, you have to put stickers of the same or opposite colour on each face. So: only green or blue stickers in F/B and only orange or red stickers in R/L. 
This is not enough though: you must also avoid diagonal corner swaps, otherwise the cube will not be solvable with the double moves group. Try to solve an N-perm with double moves only, you will give up shortly after.\\
The best thing about this technique is that once you reach HTR, the cube is very easy to solve.\\
\newline
That’s easier said than done. The downside, in fact, is that getting to HTR may require a lot of moves. On Ryan Heise's site there are some sub-optimal ways to do it. Check them out if you want to examine this technique in depth. The very important thing though, is that you are able to recognise when you are close to HTR, so you don’t miss it, rather than knowing all the possible ways to set it up. It is pretty rare to get!\\
In the following example you can see how easy the finish becomes after HTR.\\

\bigskip
\begin{tabular}{|l|}
\hline
\textbf{HTR example}\\
\hline
Scramble: \m{R\ps U\ps F D\ps F2 R2 F2 L2 D\ps U2 B2 U\ps B U2 R\ps U\ps L\ps F L F2 R\ps B R\ps U\ps F}\\
\hline
\begin{minipage}[l]{0.650\textwidth}
\bigskip
\m{F B\ps R U\ps L\ps \comment{ EO on R/L (5/5) }\\
F\ps U2 R2 F \comment{ Setup to 4c-2e (4/9) }\\
U \comment{ DR trigger, Conjugate\& solve (1/10)}\\
B\ps D2 R2 B\ps R2 B \comment{ 2e6e (6/16)}\\
\newline
Skeleton: F B\ps R U\ps L’ F\ps U2 R2 F U B\ps D2 R2 * B’ R2 B (16)\\
\newline
Now use reverse NISS at *:\\
F’ R2 L2 B\ps \comment{ HTR (4-2/ 18) } \\
R2 B2 U2 L2 F2 R2 D2 \comment{ Finish (7/25) }\\
}
\bigskip
\end{minipage}
\begin{minipage}[c]{0.25\textwidth}
\centering
\def\svgwidth{\columnwidth}
\includesvg{img/HTR1}
\end{minipage}\\
\hline
\m{Solution: F B\ps R U\ps L\ps F\ps U2 R2 F U B\ps D2 R2 F\ps R2 L2 B\ps R2 B2 U2 L2 F2 R2 D2}\\ \m{B\ps R2 B(25)}\\
\hline
\end{tabular}
\hfill \break
\hfill \break
\newline
Explanation:\\
F\ps U2 R2 F \comment{ Setup to 4c-2e }\\
\textit{F\ps U2 - setups the first column and moves the second semi-column away}\\
\textit{R2 - completes the second column}\\
\textit{F - moves the two columns on the same layer}\\
\newline
B\ps D2 R2 B\ps R2 B \comment{ 2e6e}\\
\textit{ B\ps D2 R2 - puts the corners in pairs}\\
\textit{ B\ps R2 B - solves the corners}\\
\newline
As you can see this is a simple enough case to solve. Reducing to double moves may also lead to better edge cases.\\
\newline
The former FMC 17 moves WR single by Harry Savage used HTR

\bigskip
\begin{tabular}{|l|}
\hline
\textbf{17 WR - HTR  }\\
\hline
Scramble: \m{R\ps U\ps F D\ps L2 B2 R2 B2 U F2 D U F2 R2 F D R2 B L D\ps B2 R\ps D\ps F2 R\ps}\\ \m{ U\ps F}\\
\hline
\begin{minipage}[l]{0.650\textwidth}
\bigskip
\m{R L\ps D\ps L \comment{ EO (4/4)}\\
(R2 F\ps) \comment{ DR, 2e-4c case (2/6)}\\
U\ps R2 D\ps L2 D\ps L2 D \comment{ HTR (7/13)} \\
F2 U2 F2 L2\comment{ Finish (4/17)}\\
}
\bigskip
\end{minipage}
\begin{minipage}[c]{0.25\textwidth}
\centering
\def\svgwidth{\columnwidth}
\includesvg{img/BB}
\end{minipage}\\
\hline
\m{Solution: R L\ps D\ps L U\ps R2  D\ps L2 D\ps L2 D F2 U2 F2 L2 F R2(17)}\\
\hline
\end{tabular}
\bigskip
\newpage
\hfill \break
U\ps R2 D\ps L2 D\ps L2 D \comment{ HTR }\\
\textit{ U\ps - First step towards HTR. Note that the F and R faces are “ready”. Now you must take care of corners to avoid diagonal swap cases.}\\
\textit{R2 D\ps L2 D\ps - Follow the Orange-Yellow-Green corner. these moves are a setup to put this specific corner with an orange semi-bar while move the green/blue stickers out of the way. The goal here is to make bars of opposite colours and then align them correctly}\\
\textit{ L2 D - does the job}\\
\newline
Getting an HTR simplifies the situation a lot, but sometimes it may be difficult to understand how to finish the cube right away. When you get to this state, play with it for a while: the solution must be close!\\
Also note that the “R2” in the HTR step is necessary. If you take it out you still end up with the right colour pattern but you have a diagonal corner swap because you are left with 3 corners to solve. (Do U’ D’ L2 D’ L2 D L2 and see it for yourself).\\
\newline
The last example is a solve by Chong Wen.\\

\bigskip
\begin{tabular}{|l|}
\hline
\textbf{HTR example by Chong Wen }\\
\hline
Scramble: \m{R\ps U\ps F  D\ps F2 U\ps R2 D  R2 U  L2 R  F2 U\ps B  L  R  U  L2 F2 U  R2 U\ps R\ps U\ps F }\\
\hline
\begin{minipage}[l]{0.650\textwidth}
\bigskip
\m{F (F2 U F’) \comment{ EO (4/4) }\\
(L\ps U2 D\ps L D\ps L’) \comment{ DR (6/10)}\\
(R2 U2 B2) L2 \comment{ 223 (4/14)}\\
L2 D F2 D\ps L2 B2 D\ps R2 D B2 \comment{ T perm to HTR (10-2/22)}\\
B2 D2 F2 D2 B2 \comment{ Finish (5-5/22)}\\
}
\bigskip
\end{minipage}
\begin{minipage}[c]{0.25\textwidth}
\centering
\def\svgwidth{\columnwidth}
\includesvg{img/HTR3}
\end{minipage}\\
\hline
\m{ Solution: F D F2 D\ps L2 B2 D\ps R2 D\ps F2 D2 U2 R2 L D L\ps D U2 L F U\ps F2 (22)}\\
\hline
\end{tabular}
\bigskip
\newline
Another very easy finish after HTR!



\chapter{Part Five: Tips \& Tricks}

\section{Inserting Corners}

As we have seen, the general approach to DR is to solve corners first and then fix the edges with insertions. We will talk about different ways to do this in the next section. Sometimes solving corners is not that easy though. Skeletons like 3e3c, 2e2e3c or, obviously, 3c can then be viable options to get very good results.\\
For 3 corners, there is the feeling that 8 move commutators cancel less moves on average on DR skeletons compared to regular ones, although we don't think this has been proved yet. It would be interesting to explore this topic a bit more in the future. To face this problem, here is an example of a nice 10-mover highlighted by Chong Wen:\\
\begin{center}
R2 U’ R2 B2 D’ R2 D R2 B2 U\\
B2 L2 U B2 U’ B2 L2 D’ B2 D\\
\end{center}
\bigskip
Both of these solve the same 8-mover 3 corner case in 10 moves but they seem useful especially in domino solves. Note that you can replace all the U/D moves with their wide variants to get different algorithms. These were discovered with the new  \href{https://fewestmov.es/if}{Insertion Finder} so we personally don't know if there are any other.\\
For 4+ corners, almost all skeletons are not worth it. We'll let you keep crazy ones like 10 moves to 5c). \\
A special mention goes to 2c2c skeletons. \\
The first one is an algorithm we learned from Chong Wen, the other two are from Vasco Vasconcelos:
\begin{center}
R2 U' Rw2 U F2 R2 U' Rw2 U F2 \\
 R2 L2 B2 D B2 R2 L2 F2 U F2\\
 U F2 R2 L2 B2 D B2 R2 L2 F2\\
\end{center}
These are very flexible: they can be shifted and their inverse solves the same case. You can also use the last two to solve edges as they are derived from the H perm! \\

\section{Solving Edges}

\textbf{ DR start + Corners first finish = edge insertion}\\
\newline
It’s hard to get experienced in these, but, for the most part, you will be using the same three or four edge cycles and their shifts and mirrors, on top of the ones you use for a “classic” edge skeleton.\\
\begin{center}
(R2 F2 R2 U2) *2 - edge 3-cycle\\
(R2 U2) *3 – 2e2e swap, and variations\\
(Rw2 U2 Rw2 B’) *2 – edge 5-cycle\\
\end{center}
\bigskip
We’d like to spend a few words on 2-2 edge swaps, because it’s the easiest case to find mirrors, shifts and alternate versions of.\\
\newpage
\hfill \break
Take this sequence:\\
\begin{center}
L2 D2 L2 D2 L2 D2
\end{center}
\bigskip
You can turn it in a different algorithm that may cancel more by substituting the moves in [brackets] with their opposite:\\
L2 [D2 L2] D2 [L2 D2] becomes  L2 [U2 R2] D2 [R2 U2] =  L2 U2 R2 D2 R2 U2\\
\bigskip
\newline
Here’s a brief list of some easy to learn 2e2e cases:\\
\bigskip
\newline
• Classic 2e2e (2axis) \\
L2 (D2 L2) D2 (L2 D2) =  L2 U2 R2 D2 R2 U2\\
or\\
(L2 D2) L2 (D2 L2) D2 =  R2 U2 L2 U2 R2 D2\\
or\\
(L2) D2 (L2 D2) L2 (D2) =  R2 D2 R2 U2 L2 U2 \\
\newline
• M2 2e2e\\
M2 U2 M2 U2 = L2 R2 D2 L2 R2 U2 \\
shift: (R2 D2) L2 R2 (U2 L2) =  L2 U2 L2 R2 D2 R2\\
\newline
• Strange 2e2e (3axis):\\
L2 (D2 L2 F2) D2 (L2 D2) F2 =  L2 U2 R2 B2 D2 R2 U2 F2\\
\newline
• T-shaped 2e2e (3axis)\\
D2 (L2 F2 D2) L2 (D2 F2 L2) =  D2 R2 B2 U2 L2 U2 B2 R2\\
\newline
• Slicey 2e2e + shifted versions\\
M' S2 M’ S2 = (L' R) U2 D2 (L R') F2 B2\\
shift: D2 L R' F2 B2 L' R U2\\
\newline
The following is an example on how these could be used.\\

\bigskip
\begin{tabular}{|l|}
\hline
\textbf{Edges Example}\\
\hline
Scramble: \m{R\ps U\ps F L2 F2 L2 U F2 D2 L2 D2 U\ps L2 B\ps U L\ps R U2 F D U2 L\ps B R U\ps}\\ \m{ R\ps U\ps F }\\
\hline
\begin{minipage}[l]{0.650\textwidth}
\bigskip
\m{L\ps R B \comment{ EO (3/3)}\\
(L\ps U2 L\ps) \comment{ DR trigger, reduces to 4c-1e (3/6)}\\
U L2 B2\comment{ Setup to Column \& bar (3/9)}\\
L D2 L\ps \comment{ DR trigger (3/12)}\\
U\ps L2 D R2 D2 \comment{ 6e (5/17)}\\
}
\bigskip
\end{minipage}
\begin{minipage}[c]{0.25\textwidth}
\centering
\def\svgwidth{\columnwidth}
\includesvg{img/EE}
\end{minipage}\\
\hline
\end{tabular}
\bigskip
\newline
Note that there are different ways to solve corners. Try out some of them! The best way leaves 6 edges\\
\newline
Skeleton: L\ps R B U L2 * B2 L D2 L\ps U\ps L2 D R2 D2 § L U2 L (17)\\
\newline
*: R L\ps D2 R\ps L B2 \comment{ M’ U2 M U2 edge 3 cycle (2/19) }\\
§: D2 B2 D2 L2 B2 D2 B2 L2 \comment{ Nice 2e2e swap! (5/24) }\\
\newpage
\section{Free Slice}

Inserting an algorithm is not the only way to solve edges.\\
\newline
The FMC community is going crazy about a new technique that is still under development. Since it uses slice moves to insert many edges in one or two moves in total, is commonly referred to as “slicey insertions” or “slicey shenanigans”. For the sake of simplicity, we’ll call it Free Slice (FS). The idea behind it is both simple and powerful. Take a solved cube and execute M2. The UF edge has swapped with the DB edge, the UB edge has swapped with the DF edge. If you can find a spot in your skeleton where you can insert a slice move solving 4 edges, you’ll save a lot of moves.\\
M2 also swaps the centers. This is not a problem if you know how to insert them. A center insertion will usually take up to 5 moves. Moreover, if you have more than 4 edges, you may be able to slice once to solve some and “slice back” in a different point to solve the remaining ones and the centers you offset the first time. We call slice back the insertion of the inverse of a slice previously inserted. So, if in a spot you insert S, a slice back would be an insertion of S’ further along the skeleton. With a few examples this idea will be easier to understand.\\
\newline
Center insertions are a bit tricky, I recommend reading Sebastiano’s guide to learn how to treat them.\\
\newline
Becoming fast and efficient with FS is hard and it requires a lot of practice. Here are some tips to get started:\\
-If you have 6 or 7 edges, look for a spot where a 180° slice leaves 3 or 4 edges unsolved\\
- Don’t worry about centers, you can fix them later.\\
- If you’re good at BLD, you can do a memo of the pieces and try inserting slice moves in your skeleton without stickers to speed up your insertions \\
- Enjoy yourself, be creative and study lots of example solves!\\
\newline
This section will have a lot of examples because it is one of the hardest concepts of the entire guide.\\
\newline
The first one is the current WR single by Sebastiano Tronto. If you want a visual explanation of the solve by Sebastiano himself, you can check his video: 
\url{ https://www.youtube.com/watch?v=I0yjjwxonEE}.


\bigskip
\begin{tabular}{|l|}
\hline
\textbf{Free Slice Example}\\
\hline
Scramble: \m{R\ps U\ps F D2 L2 F R2 U2 R2 B D2 L B2 D\ps B2 L\ps R\ps B D2 B U2 L U2 R\ps}\\ \m{ U\ps F }\\
\hline
\begin{minipage}[l]{0.650\textwidth}
\bigskip
\m{(U D\ps F R) \comment{ EO (4/4)}\\
(L2 F\ps B2 U\ps B2 U\ps ) \comment{ DR, 4c-1e case (6/10)}\\
(R2 B F D2) \comment{ 5e (4/14)}\\
}
\bigskip
\end{minipage}
\begin{minipage}[c]{0.25\textwidth}
\centering
\def\svgwidth{\columnwidth}
\includesvg{img/SE}
\end{minipage}\\
\hline
\end{tabular}
\bigskip
\newline
(L2 F\ps B2 U\ps B2 U\ps ) \comment{ DR, 4c-1e case (6/10)}\\
\textit{ L2 - forms a column with the E-edge\\
F\ps B2 - forms a bar with the B edge\\
U\ps B2 U\ps - DR trigger}\\
\newline
(R2 B F D2) \comment{ 5e (4/14)}\\
Pretty lucky ending: corners are solved in pairs and you just have to put them together. Note that there are different ways to do so, but R2 B F D2 is the shortest one that solves the most edges. Choose a finish that will give you a good balance of number of edges left and number of moves. 4 moves to 5e is pretty good.\\
\newline
Skeleton : (U D\ps F R L2 F\ps B2 U\ps B2 U\ps * R2 B + F D2) \\
* = U D\ps F2 D U\ps R2 \comment{2e2e (2/16) .3cycle to a 2e2e case. }\\
\newline
Take this new skeleton and look at the 4 edges left. Even without stickers, it's easy enough to spot at “+” that all 4 of them are on the same slice . With a simple “E2” move you'lll solve the 2e2e cycle in just two moves! Remember to rewrite all the moves following the slice in the new orientation of the cube, F D2 becomes B D2.\\
\newline
Skeleton: (U D\ps F R L2 F\ps B2 U\ps B2 U\ps U D\ps F2 D U\ps B + F D2) \\
+: U2 D2 // E2, FreeSlice to 4x (2/18)\\
\newline
You're now left with a center insertion.
\newline
Skeleton: (U D\ps F R L2 F\ps B2 U\ps B2 U\ps U D\ps F2 D U\ps \# B U2 D2 B D2) \\
\# = E M2 E\ps M2 // centers (2/20)\\
\newline
Once again, remember to rewrite the moves after the center insertion in the new orientation.\\
\newline
Temporary solution: D2 F\ps D2 U2 F\ps L2 R2 U\ps D B2 D B2 U B2 F L2 R\ps F\ps D U\ps (20) \\
\newline
Sebastiano then used an advanced technique to get the 16 but we'll explain it in another section.
This “lucky” solve shows the potential behind mastering DR along with CF + FS techniques. \\
\newline
The second example is a 3-edge FS solution by Tommy Kiprillis:\\

\bigskip
\begin{tabular}{|l|}
\hline
\textbf{FS Example 2}\\
\hline
Scramble: \m{R\ps U\ps F L2 U2 L2 B2 F\ps L2 R2 U2 B F2 R U L2 R2 B\ps L2 F D\ps F R F\ps R\ps }\\ \m{U\ps F}\\
\hline
\begin{minipage}[l]{0.650\textwidth}
\bigskip
\m{
(R\ps L\ps U L) \comment{ EO + square (4/4)}\\
(D2 F\ps D\ps F\ps) \comment{  222 (4/8)}\\
(B\ps U B2) \comment{  Pseudo223 (3/11)} \\
R B2 U\ps B\ps R U\ps \comment{ 3e (6/17) }\\
\newline
Skeleton: R B2 U\ps B\ps R U\ps B2 ?  U\ps B F D F D2 L\ps U\ps L R(17)\\
?: E2 \comment{(1/18) to 4e4x}\\
}
\bigskip
\end{minipage}
\begin{minipage}[c]{0.25\textwidth}
\centering
\def\svgwidth{\columnwidth}
\includesvg{img/SE2}
\end{minipage}\\
\hline
\end{tabular}
\bigskip
\newline
This shows that FS has no minimum or maximum number of unsolved edges required, apart from the physical limits of the cube obviously.\\
This first insertion solves one edge, shifts 4 centres and leaves 4 edges unsolved. You can also see this as adding a move to get a new and hopefully better skeleton: 18 moves to 4e4x.\\
\newline
Skeleton:  R B2 U\ps B\ps R U\ps B2 D2 U F B D B D2 R\ps \# U\ps R L \comment{ 18 to 4e4x}\\
\# : F E2 F\ps \comment{ Solved (4/22)}\\
\newline
Often you'll need a setup to get all 4edges on the same slice. As you can see, slicing back with E2 solves centres and edges with a single slice move.\\
This example shows the effectiveness of FS also on regular skeletons.\\
\newline
This is a 4 edges FS example by Cale Schoon:\\

\bigskip
\begin{tabular}{|l|}
\hline
\textbf{FS Example 3 }\\
\hline
Scramble: \m{ R\ps U\ps F B\ps D\ps F\ps D F2 L2 F2 U\ps B2 R2 D\ps R2 U\ps L R\ps B D R B U2 R\ps}\\ \m{U\ps F}\\
\hline
\begin{minipage}[l]{0.650\textwidth}
\bigskip
\m{
(B) U2 F \comment{ EO (3/3)}\\
D L (U B2 U2 D2) \comment{ Setup to 4c-1e(6/9)}\\
(L\ps U2 L\ps) \comment{ DR trigger (3/12)}\\
U\ps F2 R2 D R2 \comment{ 2e2e (5/17)}\\
\newline
Skeleton: U2 F D L U\ps F2 R2 D R2 L !  U2\\ L @ D2 U2 B2 U\ps B\ps \\
! = L U M2 U\ps L\ps \comment{ (6-1/22) }\\
@ = M\ps E2 M E2 = L\ps R F2 B2 L\ps R U2 D2  \comment{ (8-6/24)}\\
}
\bigskip
\end{minipage}
\begin{minipage}[c]{0.25\textwidth}
\centering
\def\svgwidth{\columnwidth}
\includesvg{img/SE3}
\end{minipage}\\
\hline
\m{Solution: U2 F D L  U\ps F2 R2 D R2 L2 U R2 L2  D' L' D2 R F2 B2 L' R B2 U' B'(24)}\\
\hline
\end{tabular}
\bigskip
\newpage
\hfill \break
Setup Explanation:\\
D L (U B2 U2 D2) \comment{ Setup to 4c-1e(6/9)}\\
\textit{ D L – simplifies\\
U B2 – forms semi-bar and E-edge column\\
D2 – forms bar\\
U2 – setups trigger}\\
\newline
This insertion is a bit difficult to see but you can follow along if you sticker your cube. Inserting the 4 edges in that way leaves 4 centres unsolved. You can insert them at @ with M\ps E2 M E2 cancelling 6 moves.\\
Free Slice peaks at multiple edges insertions.\\
\newline
This is a 6e FS example by Guido Dipietro:

\bigskip
\begin{tabular}{|l|}
\hline
\textbf{FS Example 4}\\
\hline
Scramble:\m{R\ps U\ps F L2 D R2 U2 R B\ps L2 U\ps L D L2 D L2 D L2 F2 U L2 U F2 R\ps U\ps F}\\
\hline
\begin{minipage}[l]{0.650\textwidth}
\bigskip
\m{(B L\ps U F D2 B\ps D2)  \comment{ 123 block (7/7)}\\
(F2 U\ps B2) B  \comment{ 223 (4/11)}\\
R\ps D\ps R F D F2  \comment{ 6e (6/17)}\\
\newline
Skeleton: B R\ps D\ps R * F D F2 B2 U F2 D2 B D2 F\ps U\ps ** L B\ps  \\
*: F M2 F\ps  \comment{ 6e4x different edges and shifted centres (2/19)}\\
**: M2  \comment{ 6e4x to 3e (1/20) }\\
B R\ps D\ps R F Rw2 R2 D F2 B2 U F2 D2 \$ B D2 F\ps U\ps Lw2 L\ps B\ps  \comment{ 3edges in 20}\\
\$: F L2 B F\ps D2 B\ps  \comment{ (1/21) }\\
}
\bigskip
\end{minipage}
\begin{minipage}[c]{0.25\textwidth}
\centering
\def\svgwidth{\columnwidth}
\includesvg{img/SE4}
\end{minipage}\\
\hline
\m{Solution: B R\ps D\ps R F Rw2 R2 D F2 B2 U F2 D2 F L2 B F2 U\ps Lw2 L\ps B\ps (21)}\\
\hline
\end{tabular}
\bigskip
\newline
Solving 6 edges in 4 moves is an incredible achievement.\\
\newline
This is one of my solves (Alexandros Fokianos):\\

\bigskip
\begin{tabular}{|l|}
\hline
\textbf{FS Example 5}\\
\hline
Scramble: \m{R\ps U\ps F U F2 R2 F2 D L2 U\ps F2 D2 B2 R\ps D L2 F\ps D\ps L2 R2 U\ps L2 D B2 R\ps}\\ \m{ U\ps F}\\
\hline
\begin{minipage}[l]{0.650\textwidth}
\bigskip
\m{ U2 D\ps F\comment{ EO (3/3)}\\
L\ps D\ps L\ps (F2 R2 U2) \comment{ Setup, 4c-1e case (6/9)}\\
(L U2 L\ps) \comment{ DR Trigger (3/12)}\\
(U L2 F2 U\ps R2 D B2 D) \comment{ 3e4x in 20}\\
\newline
Skeleton :U2 D\ps  F  L\ps D\ps L\ps D\ps F2 * D\ps L2 U ** B2 R2 U\ps R U2 R\ps  U2 L2 B2 \\
*  S2 (0) \comment{ 3e to 4e4x}\\
** M2 (2) \comment{ Solved}\\
}
\bigskip
\end{minipage}
\begin{minipage}[c]{0.25\textwidth}
\centering
\def\svgwidth{\columnwidth}
\includesvg{img/SE5}
\end{minipage}\\
\hline
\m{Solution: U2 D\ps F L\ps D\ps L\ps D\ps B2 U\ps R2 D R2 L2 F2 L2 U\ps L U2 L\ps U2 R2 F2 (22)}\\
\hline
\end{tabular}
\bigskip
\newline
It wasn't immediately obvious to me, but there was a 22 moves linear finish after DR:\\
(U L2 F2 L2 R2 D\ps R2 U B2 D) // finish in 22\\
\newpage
This last example by Sebastiano Tronto (second attempt at WC2019) is a good (and quite lucky) example of Free Slice insertions.\\

\bigskip
\begin{tabular}{|l|}
\hline
\textbf{FS Example 6}\\
\hline
Scramble: \m{R\ps U\ps F U2 R2 D\ps F2 D\ps L2 U\ps F2 L F\ps D\ps R\ps U\ps R2 F L2 B F R2 B2 R\ps U\ps F}\\
\hline
\begin{minipage}[l]{0.650\textwidth}
\bigskip
\m{(B D\ps R F) \comment{ EO + 3 pairs (4/4)}\\
D2 R U\ps \comment{ 2 squares (3/7)}\\
(R2 B2 R) \comment{ 3c7e (3/10)}\\
\newline
Skeleton: D2 R U\ps R\ps B2 R2 [1] F\ps R\ps D B\ps [2]\\
Insert M2 both at [1] and at [2] to leave 3e3c in 12\\
\newline
New skeleton: D2 R U\ps R\ps B2 L2 B\ps R\ps U F\ps * R2 L2\\
* = U + R U\ps R2 L2 D R\ps D\ps R2 L2 (6)\\
+ = U2 R\ps D\ps R U2 R\ps D R (6)\\
}
\bigskip
\end{minipage}
\begin{minipage}[c]{0.25\textwidth}
\centering
\def\svgwidth{\columnwidth}
\includesvg{img/SE6}
\end{minipage}\\
\hline
\m{Solution: D2 R U' R' B2 L2 B' R' U F' U' R' D' R U2 R' D R2 U' R2 L2 D}\\ \m{R' D' (24)}\\
\hline
\end{tabular}
\bigskip
\newline
Notice that the edge commutator inserted at * cancels with the M2 inserted at [2]. In fact, one can see the combination of this commutator with the free slice as a 3-move setup to a slice: U R U\ps M2 U R\ps U\ps.\\


\section{Labor Limae}

Some people from the Fewest Moves Facebook group raised the question whether it was possible to improve an already completed solve. And it is! The techniques we know of are: Replace and improve, Slice Insertions and Inserting Algorithms. Most of the times, these can be used because of block building inefficiency. World class FMC solvers usually won’t need them although they may work well for intermediate solvers.\\
The reason why we want to talk about them in this guide is that they are particularly effective when used on DR start solves because of the high number of double and opposite moves.\\
Always remember that if you don’t have much time to look for a different start, checking whether you can improve your solution can be worth the time investment, even if you are pretty confident about your skills.\\


\subsection{Replace and Improve}

The idea behind this is very simple: you take a specific part of your solution and replace it with different moves. You will get a better result if the replacing sequence is shorter or cancels some moves with your solution.\\
If you manage to do so, it usually means that your block building was not “perfect” and you can try to understand what went wrong in your skeleton.\\
This is easier said than done though. Understanding which part of the solution can be replaced is not simple. Some cubers will say things like “that part seemed inefficient” or “it didn’t smell right”. This doesn’t really help. But a tip that we can give you is: look for parts that look like domino finishes. They are not the only type of sequences this works for, but you have a better chance of succeeding.\\
\newline
Let’s look at an example to see how to do this concretely.\\
\newline
\textbf{R\ps U\ps F R2 B2 L\ps F2 D2 L\ps B2 R\ps B2 U\ps B2 D\ps F L U2 L\ps D U\ps L F R\ps U\ps F}\\
\newline
First Solution: \\
B\ps D\ps R2 F\ps L\ps R2 U L\ps B L B\ps R B (L\ps B2 R\ps B2 L R) [D U B2 D\ps B2 U’ D] (26)\\
\newline
Carefully look at the parts in brackets. They clearly look “domino-ish”.\\
First, apply L\ps B2 R\ps B2 L R to a solved cube. It's easy to see that R2 U2 R U2 R solves the cube and it's one move shorter.\\
Now replace (…) with the inverse of the sequence you found: R\ps U2 R\ps U2 R2\\
Do the same again with D U B2 D\ps B2 U\ps D. You should notice that applying L2 D L2 D2 solves the cube. You saved three more moves.\\
Now replace […] with the inverse of that sequence: D2 L2 D\ps L2\\
If you write down the new solution, it's 4 moves shorter:\\
\newline
B\ps D\ps R2 F\ps L\ps R2 U L\ps B L B\ps R B R\ps U2 R\ps U2 R2 D2 L2 D\ps L2 (22)\\
\newline
It’s now obvious that Replace and Improve does its best when applied to domino solves. 
This also means that you shouldn’t worry about always getting the optimal solution because you can fix it afterwards. \\
\newline
The most remarkable example of this technique is the current world record single. In a previous section we were left with a temporary 20 move solution.\\
\newline
\textbf{R\ps U\ps F D2 L2 F R2 U2 R2 B D2 L B2 D\ps B2 L\ps R\ps B D2 B U2 L U2 R\ps U\ps F}\\
\newline
Temporary solution: D2 F\ps D2 U2 F\ps L2 R2 {U\ps D B2 D B2 U} B2 F L2 R\ps F\ps D U\ps (20)\\
\newline
Once Sebastiano got this 20 he thought it was not worth looking for a different start. So he tried to use the Replace \& Improve technique to find a shorter solution:\\
\newline
Take the sequence in brackets.\\
Replace U\ps D B2 D B2 U with R2 D R2 D and you save four moves: two because it's shorter and two because it cancels. This is what he called, “A 20 minus 4 style points”.\\
Notice that if you take the sequence {R2 U\ps D B2 D B2 U} you could replace it with D R2 D and get the exact same result without the lucky cancellation.\\
\newline
So the final solution is: D2 F\ps D2 U2 F\ps L2 D R2 D B2 F L2 R\ps F\ps D U\ps (16)\\
\newline
After the competition, someone else found a DR that only differs from Sebastiano's by the DR trigger and it's way simpler! That's why is important to check any possible trigger after a setup!\\
\newline
(U D\ps F R) \comment{ EO (4/4)}\\
(L2 F\ps B2) \comment{ Setup (3/7)}\\
(D\ps R2 D\ps) \comment{ Variation of the DR trigger (3/10)}\\
Can you spot the 6 move ending from here?\\
\newline
Here is a solve Sebastiano showed us in which he successfully replaced a non domino-ish sequence.\\
\newline
R\ps U\ps F U2 L\ps D2 B2 L R F2 R2 D\ps R2 U\ps L\ps D\ps B2 D2 F\ps D2 R\ps U R\ps U\ps F \\
\newline
Solution: F R L U\ps B F2 D\ps F2 D F2 D2 F2 {D R\ps L D\ps R D L\ps D\ps} L\ps D L D L2 U (26) \\
But something doesn't smell right… \comment{ (see? we told you!)}\\
\newline
Scramble: {D R\ps L D\ps R D L\ps D\ps}\\
Solution: B R\ps D R D\ps B\ps\\
Updated solution: F R L U\ps B F2 D\ps F2 D F2 D2 F2 B D R\ps D\ps R B\ps L\ps D L D L2 U (24)\\
\newline
Much better!\\

\subsection{Slice insertions}

Same as before, the idea behind this is very simple: you insert a slice somewhere in the solution, leaving 4e4x unsolved, and you slice back somewhere after to solve the 4e4x. \\
We’ll borrow Chong Wen’s explanation for this since it is very clear.\\
\newline
Assuming DR is done on U/D:\\
“First we find a place with simultaneous U and D moves, because that is the only way we can reduce the solution’s length.\\
\newline
U’ L2 U ** L2 U F2 D2 R2 * D2 U’ B2\\
\newline
Every * is one of them. The next thing to do is to ignore all U and D moves and just look at the “side” moves. I we delete all of them, the string looks like this:\\
\newline
L2 ** L2 F2 R2 * B2\\
\newline
Notice that if you do R2 F2 L2 to get from ** to *, your edges simply do an E move and stay permuted relative to one another, so the E layer is unaffected and you can undo your first slice move there.\\
\newpage
\hfill \break
So with: \\
**: E' // (2-2/0)\\
*: E // (2-3/-1)\\
you can shorten your sequence.\\
\newline
Often you get easier cases like: ** F2 L2 L2 F2 * , where the slice is clearly unaffected. Sometimes you can get sequences where you can only ‘transfer’ an E2 across but not an E or E’, like: ** F2 B2 * “.\\
\newline
Here are some examples. The first two are from Chong Wen himself, the last ones are from Firstian Fushada. 1 is from a regular skeleton, 2, 3 and 4 are from DR starts.\\
\newline
1) R\ps U\ps F D2 R\ps D2 R2 U2 F2 R\ps U2 L\ps U\ps R\ps D\ps R2 B\ps F\ps U\ps L\ps F\ps R2 U R\ps U\ps F\\
\newline
F\ps D\ps R\ps F2 L R2 F\ps B\ps * L2 B R2 B\ps L2 B F R2 ** F\ps L B\ps D2 R\ps U2 R\ps F D R\ps (26)\\
*=B Bw\ps (2-3/25)\\
**=F Fw\ps (2-2/25)\\
F\ps D\ps R\ps F2 L R2 F2 z L2 B R2 B\ps L2 B F R2 Fw\ps L B\ps D2 R\ps U2 R\ps F D R’ (25)\\
\newline
2) R\ps U\ps F D2 F2 L\ps F2 L\ps B2 R D2 U2 R2 F2 D\ps R D\ps B2 D B\ps D F U\ps R\ps U\ps F \\
\newline
B2 L F2 B2 R2 D2 L\ps D R2 D R2 U\ps R2 U ** R2 B2 F2 U\ps R2 * U\ps D\ps F U\ps R2 (24)\\
*=U Uw\ps (2-3/23)\\
**=U\ps Uw (2-2/23)\\
B2 L F2 B2 R2 D2 L\ps D R2 D R2 U\ps R2 D B2 R2 L2 U\ps B2 D2 F U\ps R2 (23)\\
\newline
3) R\ps U\ps F D B2 U2 L2 F2 U B2 D L R\ps B D2 R2 F\ps U R\ps F\ps R\ps U R\ps U\ps F\\
\newline
F D R U B R L\ps D L D\ps R\ps F2 U F B L2 F B * D2 U2 R2 D\ps L2 D R2 ** (25)\\
* = E2 // 4C4X (2-4/23)\\
** = E2 // Solved (2-1/24)\\
F D R U B R L\ps D L D\ps R\ps F2 U B F L2 F B L2 D\ps R2 U2 D\ps R2 (24)\\
\newline
4) R\ps U\ps F R U R2 B\ps U2 L2 R2 B\ps R2 B\ps U2 B F2 U B\ps R2 B2 U2 R\ps U R\ps U\ps F\\
\newline
F U2 R\ps B U2 B2 R2 D U2 R\ps F2 R U R2 **** U R2 U2 R2 *** U D\ps R2 * U\ps D2 F2 D F2 ** U2 (27)\\
* = E2 (2-3/26)\\
** = E2 (2-2/26)\\
*** = E (2-4/24)\\
**** = E\ps (2-1/25)\\
F U2 R\ps B U2 B2 R2 D U2 R\ps F2 R U R2 U2 D\ps F2 U2 F2 R2 U B2 D B2 D2 (25)\\
\newline
Slice insertions can be considered a more general case of Replace and improve. \\
If the two slices are very close in the skeleton and you take apart the sequence that starts at the first slice and ends at the slice back, you should be able to find the replacement very easily. It’s the same thing, framed in a different way. With slice insertions though, you can insert the first slice at the start of the skeleton and the second one towards the end, giving you more freedom of choice than R\&I. It’s not that easy to replace a long sequence, it’s an FMC attempt itself.
As you have seen in the examples above, you can also insert multiple slices to cancel more moves, leaving the slice unsolved throughout the skeleton. This is not possible with R\&I.\\
The downside of this technique is that you won’t be able to shorten the non domino-ish sequences, as they don’t depend on slice inefficiency.\\
\newline
The following is Sebastiano’s 16 reframed using slice insertions instead of R\&I.\\
\newline
R\ps U\ps F D2 L2 F R2 U2 R2 B D2 L B2 D\ps B2 L\ps R\ps B D2 B U2 L U2 R\ps U\ps F\\
\newline
Temporary: D2 F\ps D2 U2 F\ps L2 R2 U\ps D * B2 D B2 ** U B2 F L2 R\ps F\ps D U\ps (20)\\
* : E (2-4/18)\\
**: E' (2-2/18)\\
Solution: D2 F\ps D2 U2 F\ps L2 R2 R2 D R2 D  B2 F L2 R\ps F\ps D U\ps  (18)\\
\newline
Final Solution:D2 F\ps D2 U2 F\ps L2 D R2 D  B2 F L2 R\ps F\ps D U\ps (16)\\
\newpage
\subsection{Inserting algorithms}

There two kinds of algorithms that can be inserted to shorten your solution.\\
The first ones are the “useless” algorithms, the ones that don’t change the state of the cube after they are applied.\\
\newline
A few examples are:
\begin{center}
M2 E2 M2 E2\\
M2 S2 M2 S2\\
E2 S2 E2 S2\\
\end{center}
The aim is to cancel more moves than the algorithm’s length. In this example, Walker Welch managed to do so:\\
\newline
\textbf{R\ps U\ps F U\ps F2 L2 D\ps L2 U\ps R2 D F2 L2 R\ps F\ps U F2 D B U\ps L2 D\ps L U R\ps U\ps F}\\
\newline
L2 U\ps R U R2 L D2 L\ps D R\ps D2 R2 D\ps L\ps D R\ps D\ps L R\ps D2 U2 * R2 L\ps B D U R2 B (28)\\
*: U2 D2 R2 L2 U2 D2 R2 L2 (8-9/27)\\
\newline
The second kind of algorithms you can insert are commutators. If you see that a sequence in your solution is, for example, an 8 move corner commutator, you can insert its inverse before it to cancel a whopping 8 moves (+8-16, you are basically taking out the 3c comm). Now you are left with a 3c skeleton and all you have to do is find a spot where the commutator cancels at least 1 move. This is very unlikely, but less extreme versions of this can happen. \\
Say you spot a sequence in your solution that is part of a commutator. If you insert the inverse of the full commutator, cancelling out those moves (e.g. +8-12), you are again left with a 3c skeleton, but this time the insertion has to cancel at least 5 moves to be worth it.\\
\newline
These are obviously very rare occasions so don’t expect to get them very often.\\


\section{Centers with NISS}

Sometimes, when doing CF DR, you will have the need of leaving the centres unsolved. If you also have to use NISS after you have already built some blocks around the wrong centers, you will have some problems when writing down the skeleton. Let’s look at an example.
This solve from WC17 had a premade 222 on skew centres. Sebastiano found this nice skeleton that lead to a 26.

\bigskip
\begin{tabular}{|l|}
\hline
\textbf{Centers With Niss}\\
\hline
Scramble: \m{R\ps U\ps F U R2 B2 D\ps L2 F2 D U\ps F2 R\ps B D\ps F\ps U\ps L R D F2 U F\ps R\ps U\ps F}\\
\hline
\begin{minipage}[l]{0.650\textwidth}
\bigskip
\m{(U2 L2 B2 U) \comment{ 3 pairs (4/4)}\\
R2 D2 R2 \comment{ Blocks (3/7)}\\
(L B\ps) \comment{ 2x2x3 + square (2/9)}\\
(L2 U2 L\ps U2 L\ps U2 L) \comment{ All but 3 corners (and centers) (7/16)}\\
}
\bigskip
\end{minipage}
\begin{minipage}[c]{0.25\textwidth}
\centering
\def\svgwidth{\columnwidth}
\includesvg{img/NC}
\end{minipage}\\
\hline
\end{tabular}
\bigskip
\newline
The solve is pretty standard up to this point. Now Sebastiano went on to write the skeleton for the inverse scramble.\\
\newline
Skeleton: (U2 L2 B2 U L B\ps L2 U2 L\ps U2 L\ps U2 L B2 L2 B2)\\
\newline
But wait, shouldn’t the last three moves be R2 D2 R2? Think about NISS as a long sequence of moves that doesn’t do anything to the cube. If the centers are shifted you have to imagine inserting the center commutator in-between the moves on the normal scramble and the moves on the inverse scramble. When the moves done on inverse become premoves (to be put at the end of the skeleton), they are done after the centers are corrected. This obviously works also if you write down the skeleton on inverse, using the moves on the normal scramble as premoves. Our suggestion is to make a small grid with the moves before and after the center insertion. In this case it would be:\\
\bigskip
\begin{center}
U | B\\
D | F\\
F | L\\
B | R\\
R | U\\
L | D\\
\end{center}
Now all you have to do is “translate” the moves according to the grid. But here comes the tricky part. If you are writing the skeleton for the normal scramble, the grid has to be read left to right: “before | after”. If, as in this case, you are writing the skeleton for the inverse scramble, the grid has to be read right to left: “after | before”.\\
\newline
Going back to the solve, the moves at the end of the skeleton are, in fact, “translated” from R2 D2 R2 to B2 L2 B2. Now you can go on with insertions.\\

\chapter{Some final words}

\section{The Authors}

That's all folks! We wish that you had as much fun reading the guide as we had writing it. Hopefully everything was clear enough for you to understand. 
Once again, we'd like to thank Sebastiano for his precious help. Having the opportunity to add to what has become a cornerstone of the Fewest Moves community was unbelievable. We hope we stood up to the test. We also want to say thank you to the italian FMC group, \textbf{"Quelli troppo lenti per lo speedsolving"} ("The ones too slow for speedsolving"), which kept us motivated over the past months. Probably we wouldn't even have thought of writing this if it wasn't for you. \\
Ok, now wipe your tears off, we must go on.\\
\newline
First of all we want to give credits to all the people that were cited in this guide.\\
Here's the full cast:\\
\newline
• The puppet master: Sebastiano Tronto \href{https://www.worldcubeassociation.org/persons/2011TRON02}{[2011TRON02]}\\
\newline
• The authors: Alexandros Fokianos \href{https://www.worldcubeassociation.org/persons/2017FOKI01}{[2017FOKI01]}, Tommaso Raposio \href{https://www.worldcubeassociation.org/persons/2014RAPO01}{[2014RAPO01]}\\
\newline
• The actors, in order of appearence:\\
\newline
Harry Savage \href{https://www.worldcubeassociation.org/persons/2013SAVA01}{[2013SAVA01]} \\
Chong Wen \href{https://www.worldcubeassociation.org/persons/2014WENW01}{[2014WENW01]} \\
Alexandre Campos \href{https://www.worldcubeassociation.org/persons/2015CAMP17}{[2015CAMP17]}\\
Attila Horváth \href{https://www.worldcubeassociation.org/persons/2012HORV01}{[2012HORV01]}\\
Vasco Vasconcelos \href{https://www.worldcubeassociation.org/persons/2008VASC01}{[2008VASC01]}\\
Tommy Kiprillis \href{https://www.worldcubeassociation.org/persons/2014KIPR01}{[2014KIPR01]}\\
Cale Schoon \href{https://www.worldcubeassociation.org/persons/2014SCHO02}{[2014SCHO02]}\\
Guido Dipietro \href{https://www.worldcubeassociation.org/persons/2013DIPI01}{[2013DIPI01]}\\
Firstian Fushada \href{https://www.worldcubeassociation.org/persons/2015FUSH01}{[2015FUSH01]}\\
Walker Welch \href{https://www.worldcubeassociation.org/persons/2011WELC01} {[2011WELC01]}\\
\newline
We assumed you are ok with us borrowing your solves (and in some case explanations) because you posted them on Facebook. If this is not the case, or if there is anything that bothers you, please tell us. \\
We also want to stress how important the Facebook group "Fewest Moves" was in the writing of the guide. Most of the stuff we learned comes from there originally. We then expanded through experimentation and other sources. So, even if your solve was not used as an example or you were not cited, probably we read some of your content and it contributed to the making of the project. It's something from the community, to the community.\\
\newline
Adding to this, if you think that something important was not covered, don't worry! We plan on making a version 2.0, expanding the present content and perfecting the appearance: maybe some new triggers get discovered? maybe we missed some? Also there might be some mistakes or typos that we missed. Let us know! We created this email account to gather all your suggestions, critiques and spotted typos: dominoreducitionguide@gmail.com\\
\newline
We hope this is not the final version, as it means there is a lot of stuff out there still to discover.\\
\bigskip
\hfill Alexandros Fokianos \& Tommaso Raposio

\section{Sebastiano Tronto}

\textit{One of the things that holds back the advancement of cubing the most is certainly the widespread belief that some technique might be too hard to be used effectively and consistently, or perhaps just not worth the effort.\\
The Domino Reduction method (DR) for FMC was considered one of those techniques that happened to give incredibly good results from time to time, but it was deemed very inconsistent. I personally only used it when I came across a very easy or lucky reduction step, but I didn't look for it often.\\
But at the beginning of 2019 things began to change. After three people (Harry Savage, Robert Yau and myself) got a sub-20 with similar DR solves on the same scramble, the method gained popularity. Some people - Wong Chong Wen and Cale Schoon just to mention a few - decided to practice this method more consistently, often making their solves public. The results of these pioneers were promising: consistent ways to reach the DR state were found, and finishing a solve in one hour did not seem to be a problem. The viability of this method started to be reconsidered by the FMC community.\\
In June of the same year, at FMC 2019, another world record single was set using DR. But this time it wasn't just a single solve: many other great results were achieved with DR at this competition. At this point even more people started practicing DR, and two of them decided to collect what they learnt into this document. Hopefully this tutorial will help more people understand that DR is not as hard as the lack of good reference made it seem in the past.\\
I believe there are many things yet to be understood or discovered about DR. Many questions still do not have an answer: what is the most efficient method to reach the DR state? Is there any reasonably easy way to find a "direct DR", without going through EO or CO first? And after that, what is the best way to finish a solve? Solving corners first and then edges with insertions, reducing to half turns, or something completely new? Is there any way to consistently find a "direct solution", without intermediate substeps?\\
This is only the beginning of our journey around DR. Could this method be the Holy Grail of FMC, leading us to consistent sub-25 and beyond? Maybe we will soon find out.}\\
\bigskip
\hfill Sebastiano Tronto




















































\end{document}